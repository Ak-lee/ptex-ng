%!TeX Program=pTeX-ng(LaTeX)
\special{pdf: mapline + upjisr-h UniGB-UTF16-H FandolSong-Regular}
\special{pdf: mapline + upjisg-h UniGB-UTF16-H FandolHei-Regular}
\special{pdf: mapline + upjisr-v UniGB-UTF16-V FandolSong-Regular}
\special{pdf: mapline + upjisg-v UniGB-UTF16-V FandolHei-Regular}
%
\documentclass[dvipdfmx]{beamer}
\usepackage{hologo}
\usetheme{EastLansing}
\ybaselineshift.5pt
\pdfpagewidth\paperwidth
\pdfpageheight\paperheight
\begin{document}
\title{\bf 关于p\TeX-ng的零散文档}
\author{马起园}
\date{2014年11月}
\begin{frame}
\maketitle
\end{frame}
%
\parskip.5zw
\begin{frame}[fragile]
\frametitle{\bf p\TeX-ng是什么}
p\TeX-ng是下一代的p\TeX。在底层引擎上支持汉字处理、禁则处理、汉字和西文间距处理、汉字直排。目前只支持UTF-8编码,覆盖有中日韩三种语言支持。

p\TeX 是日本{\font\t=ascii10 \t ASCII}公司开发的汉字处理引擎。在2008年和2010年经历合并以及收购之后,p\TeX 的实际开发已经停止,已经转向由社区维护。

p\TeX-ng的开发最早可以追溯到2012年春节假期,最早在Lua\TeX 进行试验性质的修改,在2013年一度中断开发,正式版在今年(2014年)10月份开始发布。由于作者的时间有限,目前只做漏洞修补,下一部分大规模开发要拖到2015年春节。p\TeX-ng是自由软件,分发遵循GPL第二版许可。
\end{frame}
%
\begin{frame}[fragile]
\frametitle{\bf p\TeX-ng的开发概要}
p\TeX-ng的前身是Y\&Y \TeX,一个用C语言实现的带内存管理的\TeX。早期的Y\&Y \TeX 的源代码
是使用web2c经Pascal转换为C语言,不具有可读性和可扩展性,这部分代码在2014年上半年进行
了重写。p\TeX-ng的源代码建立在Y\&Y \TeX 的基础之上。

p\TeX-ng的的汉字处理相关代码来源于p\TeX;Unicode编码处理代码来源于up\TeX。对于\hologo{eTeX}
的支持来源于eup\TeX。

p\TeX-ng只支持PDF文件输出,不支持DVI文件输出。输出PDF相关的代码来源于dvipdfmx,没有任何
来自\hologo{pdfTeX}的代码,所以大量primitive不被支持。
\end{frame}
%
\begin{frame}[fragile]
\frametitle{\bf p\TeX-ng的将来}
p\TeX-ng在将来会支持OpenType字体、绑定动态语言扩展宏编程。
\end{frame}
%
\begin{frame}[fragile]
\frametitle{\bf p\TeX-ng的primitive}
\TeX 中的primitive是最基本的命令,宏展开之后就是primitive的一个序列。
\end{frame}
\end{document}
